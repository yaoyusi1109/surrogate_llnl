\pdfminorversion=4
\documentclass[11pt]{article}
\usepackage{hyperref}
\usepackage{amsmath, amsfonts, amssymb, mathrsfs}
\usepackage{dcolumn}
\usepackage{caption}
\usepackage{subcaption}
\usepackage{filemod}
\usepackage{natbib}
\usepackage[ruled, vlined]{algorithm2e}
\usepackage{floatrow}
\usepackage{setspace}
\usepackage{verbatim}
\usepackage{graphicx}

\hypersetup{
    colorlinks=true,
    linkcolor=blue, 
    urlcolor=black,
    citecolor=blue, 
    }

\oddsidemargin=0.25in
\evensidemargin=0.25in
\textwidth=7in
\textheight=8.75in
\topmargin=-.5in
\addtolength{\oddsidemargin}{-.5in}
\addtolength{\evensidemargin}{-.5in}
\footskip=0.5in
\doublespacing

\title{\vspace{-0.3cm} Variable Selection with Deep Gaussian Process Surrogates}
\author{Annie S. Booth\thanks{Corresponding author: Department of Statistics, 
	Virginia Tech, {\tt annie\_booth@vt.edu}} 
	\and Kevin Quinlan\thanks{Lawrence Livermore National Laboratory} 
	\and Laura Wendelberger\footnotemark[2]}
\date{\today}

\begin{document}

\vspace{-0.5cm}
\maketitle

\singlespacing

\vspace{-1.5cm}
\begin{abstract} 
TODO
\end{abstract}

\noindent \textbf{Keywords:} TODO

\doublespacing

\section{Introduction}\label{sec:intro}

Motivate - variable selection

Introduce DGPs

Competing methods: blind kriging and Zhang paper, paper that Jon sent?

\section{Review: Deep Gaussian processes}

Review classic DGPs

\section{Method}

Upgrades to monotonic warping functionality
Upgrades to allow $\tau^2_w$ to VARY (Bayesian framework with noninformative prior)

\subsection{The decision rule}

Based on the range of W

\section{Synthetic Benchmarks}

G function with 2 important, 2 not important

Tray function with 2 important, 3 not important

Larger example???

\section{Motivating Example}

10 dimensions?

\section{Discussion}

\subsection*{Acknowledgements}

Support from Lawrence Livermore National Laboratory

\singlespacing
\bibliographystyle{../jasa}
\bibliography{main}

\newpage
\setcounter{page}{1}
\begin{center}
{\large\bf SUPPLEMENTARY MATERIAL}
\end{center}
\appendix

\end{document}
